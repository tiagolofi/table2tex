% pacotes necessários
\usepackage[brazil]{babel} % ajustando idioma para pt-br
\usepackage{graphicx} % lib pra controlar altura e largura das coisas
\usepackage[table]{xcolor} % lib pra editar a cor da tabela
\usepackage{booktabs} % lib para dividir a tabela
\usepackage{caption} % pra mexer com os rodapés

% configurações dos rodapés
\captionsetup{font=footnotesize} 
\captionsetup[table]{name=Notas}

\begin{table}[h!]
\centering
\rowcolors{2}{gray!15}{white}
\resizebox{\textwidth}{!}{%
\begin{tabular}{ccccc}
\midrule
\textbf{Tratamento} &
\textbf{Repolho (kg/m²)} &
\textbf{Cebolinha (kg/m²)} &
\textbf{Rabanete (kg/m²)} &
\textbf{IEA} \\
\midrule
Re        & 4,9 & --  & --  & 1   \\
Ce        & --  & 2,4 & --  & 1   \\
Ra        & --  & --  & 3,3 & 1   \\
Re/Ce     & 4,1 & 1,1 & --  & 1,3 \\
Re/Ra     & 5,0 & --  & 1,2 & 1,4 \\
Ce/Ra     & --  & 1,1 & 1,3 & 0,8 \\
Re/Ce/Ra  & 4,2 & 0,3 & 0,7 & 1,2 \\
\midrule
\end{tabular}
}
\caption{
Re: monocultura de repolho; Ce: monocultura de cebolinha; Ra: monocultura de rabanete; Re/Ce: consórcio repolho e cebolinha; Re/Ra: consórcio repolho e rabanete; Ce/Ra: consórcio cebolinha e rabanete; Re/Ce/Ra: consórcio repolho, cebolinha e rabanete; IEA: índice de equivalência de área.
}
\end{table}
